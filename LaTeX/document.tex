\documentclass[12pt,a4paper]{amsart}
\usepackage{amsfonts}
\usepackage{amsthm}
\usepackage{amsmath}
\usepackage{amscd}
\usepackage[latin2]{inputenc}
\usepackage{t1enc}
\usepackage[mathscr]{eucal}
\usepackage{indentfirst}
\usepackage{graphicx}
\usepackage{graphics}
\numberwithin{equation}{section}
\usepackage[margin=2.9cm]{geometry}
\usepackage{epstopdf}

\def\numset#1{{\\mathbb #1}}

\theoremstyle{plain}
\newtheorem{Th}{Theorem}[section]
\newtheorem{Lemma}[Th]{Lemma}
\newtheorem{Cor}[Th]{Corollary}
\newtheorem{Prop}[Th]{Proposition}

\theoremstyle{definition}
\newtheorem{Def}[Th]{Definition}
\newtheorem{Conj}[Th]{Conjecture}
\newtheorem{Rem}[Th]{Remark}
\newtheorem{?}[Th]{Problem}
\newtheorem{Ex}[Th]{Example}

\newcommand{\im}{\operatorname{im}}
\newcommand{\Hom}{{\rm{Hom}}}
\newcommand{\diam}{{\rm{diam}}}
\newcommand{\ovl}{\overline}
%\newcommand{\M}{\mathbb{M}}

\author{Nathan Levett}
\title{Collatz musings}

\begin{document}
	
\tableofcontents

\section{Preface}
	Yeet

\section{Parameterised Function}

The Collatz function can be parameterised in such a way as; (with the interchangable use of $k \in \left ( a\:\mathrm{mod}\:b \right )$ and $k \equiv a \left ( \mathrm{mod}\:b \right )$) 

\begin{Def} The Collatz function
\begin{equation}
f\left ( n \right )=\left\{\begin{matrix}
n/2  & n \in \left ( 0\:\mathrm{mod}\:2 \right ) \\ 
3n+1 & n \in \left ( 1\:\mathrm{mod}\:2 \right )
\end{matrix}\right.
\end{equation}
\end{Def}

We can generalise this by parameterising it in the following way, maintaining that the function's deciding factor will be that the input is either $k \in \left ( 0\:\mathrm{mod}\:X \right )$ or $k \notin \left ( 0\:\mathrm{mod}\:X \right )$, and that the resulting behaviour is either a division, or a multiplication plus addition.

\begin{Def} The Parameterised Collatz function
\begin{equation}
f_{\left (P_{1},P_{2},h\right )}\left ( n \right )=\left\{\begin{matrix}
n/P_{1}  & n \in \left ( 0\:\mathrm{mod}\:P_{1} \right ) \\ 
P_{2}n+h & n \notin \left ( 0\:\mathrm{mod}\:P_{1} \right )
\end{matrix}\right.
\end{equation}
\end{Def}


\section{A new representation}
	Yeet

\end{document}