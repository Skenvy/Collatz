\documentclass[12pt,a4paper]{amsart}
\usepackage{amsfonts}
\usepackage{amsthm}
\usepackage{amsmath}
\usepackage{amscd}
\usepackage{amssymb}
\usepackage[latin2]{inputenc}
\usepackage{t1enc}
\usepackage[mathscr]{eucal}
\usepackage{indentfirst}
\usepackage{graphicx}
\usepackage{graphics}
\numberwithin{equation}{section}
\usepackage[margin=2.9cm]{geometry}
\usepackage{epstopdf}

\def\numset#1{{\\mathbb #1}}

\theoremstyle{plain}
\newtheorem{Th}{Theorem}[section]
\newtheorem{Lemma}[Th]{Lemma}
\newtheorem{Cor}[Th]{Corollary}
\newtheorem{Prop}[Th]{Proposition}

\theoremstyle{definition}
\newtheorem{Def}[Th]{Definition}
\newtheorem{Conj}[Th]{Conjecture}
\newtheorem{Rem}[Th]{Remark}
\newtheorem{?}[Th]{Problem}
\newtheorem{Ex}[Th]{Example}

\newcommand{\im}{\operatorname{im}}
\newcommand{\Hom}{{\rm{Hom}}}
\newcommand{\diam}{{\rm{diam}}}
\newcommand{\ovl}{\overline}
%\newcommand{\M}{\mathbb{M}}

\author{Nathan Levett}
\title{Collatz musings}

\begin{document}
	
\tableofcontents

\section{Preface}

A collection of observations on the Collatz conjecture, with no aim to "solve" it so much as to waste time on a seemingly unsolvable problem. There are two main directions that this goes in. First is a look at what can be said about a generalised/parameterised function, secondly on a minified/new representation of the problem.

\section{Parameterised Function}

The Collatz function can be parameterised in such a way as; (with the interchangable use of $k \in \left ( a\:\mathrm{mod}\:b \right )$ and $k \equiv a \left ( \mathrm{mod}\:b \right )$) 

\begin{Def}[The Collatz function]
\begin{equation}
f\left ( n \right )=\left\{\begin{matrix}
n/2  & n \in \left ( 0\:\mathrm{mod}\:2 \right ) \\ 
3n+1 & n \in \left ( 1\:\mathrm{mod}\:2 \right )
\end{matrix}\right.
\end{equation}
\begin{equation}
f^{k}\left ( n \right )=\left\{\begin{matrix}
n & k=0 \\ 
f\left ( f^{k-1}\left ( n \right ) \right ) & k \geq 1 )
\end{matrix}\right.
\end{equation}
\end{Def}

We can generalise this by parameterising it in the following way, maintaining that the function's deciding factor will be that the input is either $k \in \left ( 0\:\mathrm{mod}\:X \right )$ or $k \notin \left ( 0\:\mathrm{mod}\:X \right )$, and that the resulting behaviour is either a division (by the whatever $X$ happens to be), or a multiplication plus addition.

\begin{Def}[The Parameterised Collatz function]
\begin{equation}
f_{\left (P_{1},P_{2},h\right )}\left ( n \right )=\left\{\begin{matrix}
n/P_{1}  & n \in \left ( 0\:\mathrm{mod}\:P_{1} \right ) \\ 
P_{2}n+h & n \notin \left ( 0\:\mathrm{mod}\:P_{1} \right )
\end{matrix}\right.
\end{equation}
\end{Def}

The primary goal of having a parameterised version of the Collatz function is to evaluate what constraints can be placed on the parameters such that they cannot be demonstrated to break the requirements of;

\begin{Conj}[The Collatz Conjecture]
\begin{equation}
\forall n \in \mathbb{N}^{1} \exists \: m \in \mathbb{N}^{1} \rightarrow f^{m}\left ( n \right )=1
\end{equation}
\end{Conj}

To evaluate which parameters fail to adhere to the conjecture, there are some specific behaviours of the function we can check against. The first and most obvious is whether or not some parameters will void the intent of the conjecture (that all starting values will eventually result in 1) by trapping specific, or all, starting values in an infinite growth. The other behaviour is an emergent property of;

\begin{Conj}[The uniqueness of the cycle $ \left ( 1\rightarrow 4\rightarrow 2\rightarrow 1\rightarrow * \right ) $.]
\begin{equation}
\forall n \in \left\{\begin{matrix}
\left ( \mathbb{N}^{1}/\left \{ 1,2,4 \right \} \right ) \nexists \: m \in \mathbb{N}^{1} \rightarrow f^{m}\left ( n \right )=n \\ 
\left \{ 1,2,4 \right \} \rightarrow f^{3}\left ( n \right )=n
\end{matrix} \right.
\end{equation}
\end{Conj}

If every starting value must eventually reach 1, no starting value can cause an iteration of the function to enter a cycle until the iteration has reached 1, at which point it will enter the looping iteration of what is conjectured to be the unique cycle of $ \left ( 1\rightarrow 4\rightarrow 2\rightarrow 1\rightarrow * \right ) $. Whilst the generic statement of the conjecture does not need any reworking for the now generalised form of the equation, obviously the "uniqueness of the cycle" does!

\begin{Conj}[The uniqueness of the cycle starting from 1]
\begin{equation}
\begin{matrix}
\mathrm{C}_{b} = \left \{ f^{a} \left ( 1 \right ) | 1 \leq a \leq b | f^{b} \left ( 1 \right ) = 1 | 0 \neq c < b \rightarrow f^{c} \left ( 1 \right ) \neq 1 \right \} \\
\forall n \in \left\{\begin{matrix}
\left ( \mathbb{N}^{1}/\mathrm{C}_{b} \right ) \nexists \: m \in \mathbb{N}^{1} \rightarrow f^{b}\left ( n \right )=n \\ 
\mathrm{C}_{b} \rightarrow f^{b}\left ( n \right )=n
\end{matrix} \right.
\end{matrix}
\end{equation}
\end{Conj}

\subsection{Values of n that preempt descent} \hfill\\

A

\subsection{Cycle starting from 1} \hfill\\

A

\section{A new representation}

\subsection{Reducing the conjecture to values that are in (4 mod 6)} \hfill\\

Our motivation is to try to find some way to "speed up" travelling from any $n$ to $1$. Can we identify certain numbers that act as hotspots for more extended branches in the Collatz graph? While not the main focus of this, we begin by looking at;

\begin{Def}[The Inverse Collatz function]
\begin{equation}
f^{-1} \left ( n \right )=\left\{\begin{matrix}
\left \{ 2n \right \} & n \notin \left ( 4\:\mathrm{mod}\:6 \right ) \\ 
\left \{ 2n, \frac{n-1}{3} \right \} & n \in \left ( 4\:\mathrm{mod}\:6 \right )
\end{matrix}\right.
\end{equation}
\end{Def}

Every number can be reached simply by a previous iteration of the function halving whatever is double that value. Thus no matter the value, the inverse Collatz function would include double the starting value. To figure out which values of $n$ could be the result of the previous iteration taking the $3n+1$ route is a little more involved. To have taken the $3n+1$ route, the value $n$ (in the previous iteration) must have been odd ($n \in \left ( 1\:\mathrm{mod}\:2 \right )$) thus making the resulting value of $3n+1$ an even number. Therefore it is not enough to simply require (in the current iteration) ($n \in \left ( 1\:\mathrm{mod}\:3 \right )$), but also that $n$ is even ($n \in \left ( 0\:\mathrm{mod}\:2 \right )$ (for if $n$ were odd instead and the result of applying $3n+1$, then the previous iteration's $n$ would have had to be even, thus would not have taken the $3n+1$ route). We can take the intersection these requirements ($n \in \left ( 0\:\mathrm{mod}\:2 \right )$) and ($n \in \left ( 1\:\mathrm{mod}\:3 \right )$) to get the requirement ($n \in \left ( 4\:\mathrm{mod}\:6 \right )$) for isolating values of $n$ that could have resulted from the previous iteration applying the $3n+1$ rule. Hence, when ($n \in \left ( 4\:\mathrm{mod}\:6 \right )$), the inverse Collatz function must include the $\frac{n-1}{3}$ value.

\begin{Def}[Equivalent statement of the conjecture]
It is equivalent to the originally stated Collatz Conjecture to conjecture that all values in the residue class $\left ( 4\:\mathrm{mod}\:6 \right )$ will eventually reach the value $4$
\end{Def}

Starting from any number, consecutive iterations of the Collatz function will repeatedly remove multiplicities of $2$ until the result is no longer integer divisible by $2$, at which point it will apply the $3n+1$ step, resulting in a number $n$ that is ($n \in \left ( 4\:\mathrm{mod}\:6 \right )$). This number will then follow the same pattern of iteratively removing multiplicites of $2$ until it again is no longer even, at which point it will once again go through the $3n+1$ step ang yield another $n$ that satisfies ($n \in \left ( 4\:\mathrm{mod}\:6 \right )$). Knowing that every starting number will eventually, or immediately, yield a number that is ($n \in \left ( 4\:\mathrm{mod}\:6 \right )$) means we can instead focus on how values in the residue class $\left ( 4\:\mathrm{mod}\:6 \right )$ traverse from one to the next, while treating $4$ as their final destination as $4$ is the $3n+1$ result of starting from 1!

While there are many ways to reduce the effort of "proof" for the Collatz conjecture by presenting equivalent statements to the conjecture to minify the set of numbers to "prove" something for, there are two we will use here. The first is simply reducing the conjecture to instead be;

\begin{Conj}[The Minified Collatz Conjecture]
\begin{equation}
\forall n \in \mathbb{N}^{1} \exists \: m \in \mathbb{N}^{1} \rightarrow f^{m}\left ( n \right )<n
\end{equation}
\end{Conj}

Which has yet another possible reduction (by observing that this is already true for any starting even number, we can restrict the necessity of the conjecture to only stating it for odd number;

\begin{Conj}[The 2*Minified Collatz Conjecture]
\begin{equation}
\forall n>1\mathrm{,}\:n \in \left ( 1\:\mathrm{mod}\:2 \right ) \exists \: m \in \mathbb{N}^{1} \rightarrow f^{m}\left ( n \right )<n
\end{equation}
\end{Conj}

Here is where we can apply the next reduction that observes the result of the function on an odd number will be ($n \in \left ( 4\:\mathrm{mod}\:6 \right )$)

\begin{Conj}[The 3*Minified Collatz Conjecture]
\begin{equation}
\forall n\geq4\mathrm{,}\:n \in \left ( 4\:\mathrm{mod}\:6 \right ) \exists \: m \in \mathbb{N}^{1} \rightarrow f^{m}\left ( n \right )<n
\end{equation}
\end{Conj}

While this is as far as we "need" to minify the conjecture at this point, we will go one step further to fully embrace that we are anchoring the jumps between values after multiple iterations to the ($n \in \left ( 4\:\mathrm{mod}\:6 \right )$) values of $n$.

\begin{Conj}[The 4*Minified Collatz Conjecture]
\begin{equation}
\forall n>4\mathrm{,}\:n \in \left ( 4\:\mathrm{mod}\:6 \right ) \exists \: m \in \mathbb{N}^{1} \rightarrow f^{m}\left ( n \right )<n\mathrm{,}\:f^{m}\left ( n \right ) \in \left ( 4\:\mathrm{mod}\:6 \right )
\end{equation}
\end{Conj}

\subsection{Jumps between ($n \in \left ( 4\:\mathrm{mod}\:6 \right )$)} \hfill\\

We will now explore how we can expand the numbers in the residue class $\left ( 4\:\mathrm{mod}\:6 \right )$ to sub residue classes that partition their super residue classes. For example, and pertinent to the next section this is;

\begin{Def}[Residue classes and their equivalent partitioning]
If we say that a residue class $\left ( a\:\mathrm{mod}\:b \right )$ is a (super)* residue class, we refer to it's capacity to be partitioned into the (sub)* residue classes $\left ( \left (a+db  \right )\:\mathrm{mod}\:\left (cb  \right ) \right )$ such that;
\begin{equation}
\forall \:a\mathrm{,}\:b\mathrm{,}\:c \in \mathbb{N}^{1} \rightarrow \left ( a\:\mathrm{mod}\:b \right ) \equiv \left (\bigcup_{d=0}^{c-1} \left \{ \left (a+db  \right )\:\mathrm{mod}\:\left (cb  \right ) \right \}  \right )
\end{equation}
\end{Def}

In particular, in this section we rely on the following partitioning; 

\begin{Ex}[Partitioning $\left ( 4\:\mathrm{mod}\:6 \right )$ into 4 sub residue classes]
\begin{equation}
\left ( 4\:\mathrm{mod}\:6 \right ) \equiv \left (\bigcup_{d=0}^{3} \left \{ \left (4+d*6  \right )\:\mathrm{mod}\:24 \right \}  \right )
\end{equation}
\end{Ex}

Such that we can demonstrate the following for each of the residue classes $4$, $10$, $16$, and $22$, mod $24$.


\end{document}
